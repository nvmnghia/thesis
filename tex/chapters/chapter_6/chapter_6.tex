\providecommand{\main}{../..}

\documentclass[../../thesis]{subfiles}


\begin{document}

\chapter{Kết luận}\label{chap:conclusion}

\section{Tổng kết nội dung}\label{sec:summary}

Khóa luận mang tên "Ứng dụng đọc truyện tranh". Do vô tâm, khóa luận không được
sửa thành "yacv: Ứng dụng đọc truyện tranh ngoại tuyến trên Android" khi tên
yacv được quyết định và có cơ hội sửa. Nội dung của khóa luận là xây dựng một
ứng dụng trên Android giúp đọc các tệp truyện có sẵn trên điện thoại.

Các chức năng của ứng dụng được xây dựng theo một nhóm nhỏ người dùng say mê
(enthusiast). Chức năng chính của ứng dụng gồm xem truyện, xem metadata truyện,
và tìm theo metadata.

\section{Kết quả đạt được}\label{sec:thesis_result}

Ứng dụng thực hiện được một số kết quả sau:

\begin{itemize}
    \item Có đủ ba chức năng cơ bản
    \item Có chức năng xóa truyện, Yêu thích truyện
    \item Tốc độ cuộn trang chấp nhận được
\end{itemize}

Tuy nhiên, ứng dụng chưa thực hiện được một số mục tiêu cá nhân:

\begin{itemize}
    \item Hỗ trợ nhiều loại tệp truyện
        \item CBR (tệp nén RAR), CBT (tệp nén RAR),\ldots
        \item Hỗ trợ tệp nén đặc (xem giải thích ở cuối \autoref{sec:zip})
    \item Hỗ trợ nhiều loại metadata
    \item Test bao phủ 80\%
    \item Chức năng xem trước trang truyện
    \item Liên kết với một số nguồn thông tin truyện tranh như
        GrandComicDatabase
\end{itemize}

\section{Định hướng phát triển}\label{sec:future-work}

Do nhắm đến một tập người dùng nhỏ, ứng dụng có thể xếp vào loại thị trường
ngách (niche market). Người dùng ở thị trường này khá trung thành, do đó bản
thân họ không cần tính năng nào mới mà cần độ hoàn thiện cho sản phẩm.

Tuy nhiên, vẫn có hai tính năng mới đáng kể có thể được đưa vào ứng dụng, nếu
xét đến các vấn đề kĩ thuật liên quan thay vì nhu cầu thực của người dùng:

\begin{itemize}
    \item
        Tự động tải truyện:

        Các trang web truyện cụ thể có thể được hỗ trợ trong ứng dụng, hoặc phát
        triển một bộ lấy dữ liệu (scraper) có khả năng tự động phát hiện ảnh
        trong một trang web và tải về.

        Truyện khi tải về có thể được đóng gói thành định dạng CBZ.
    \item
        Tăng chất lượng truyện:

        Đây là tính năng bổ sung cho tính năng trên, do nhu cầu cốt yếu của
        người dùng là đọc tệp truyện chất lượng cao.

        Một hướng phát triển cho tính năng này là dùng Tensorflow Lite để nhúng
        các mô hình học máy giúp tăng chất lượng ảnh như ESRGAN hay đặc biệt là
        SRCNN - tối ưu cho ảnh anime - vào ứng dụng.
\end{itemize}

\section{Đánh giá cá nhân}\label{sec:personal-assessment}

Toàn bộ khóa luận được hoàn thành trễ ba tháng so với kì vọng ban đầu là xong mã
nguồn trước Tết Âm lịch Tân Sửu 2021, và không thực hiện được bất kì mục tiêu cá
nhân nào.

Cách thức làm việc trong thời gian thực hiện khóa luận cũng khiến thầy giáo
hướng dẫn và bản thân em không hài lòng, dù vậy vẫn không chịu thay đổi. So với
thời điểm thực tập ở Viettel hai năm trước, đây là sau tháng bước lùi.

\end{document}
