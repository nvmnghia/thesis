\documentclass[../../thesis]{subfiles}

\IfEq{\jobname}{\detokenize{thesis}}{}{%
    \setcounterref{chapter}{chap:intro}
    \addtocounter{chapter}{-1}
}


\begin{document}

\chapter{Giới thiệu}\label{chap:intro}

%----------------------------------------------------------------------------------------
%	1.1: Đặt vấn đề
%----------------------------------------------------------------------------------------

\section{Đặt vấn đề}\label{sec:background}

Tại Đông Á và Đông Nam Á, văn hóa truyện tranh gốc Á, nhất là truyện tranh Nhật
(manga), được đón nhận khá tích cực, đặc biệt trong giới trẻ. Thế hệ những người
dưới 40 tuổi hiện nay được tiếp xúc với truyện tranh từ sớm, thông qua những
cuốn truyện truyền tay và phim hoạt hình dựa trên truyện tranh, và tiếp tục đọc
dù đã qua tuổi thiếu niên. Một số tác phẩm manga còn có lượng người đọc lớn trên
toàn cầu như Doraemon, One Piece. Ở bên kia bán cầu, với sự thành công của vũ
trụ điện ảnh Marvel và DC, truyện tranh phương Tây (comic) cũng được hồi sinh
phần nào sau một thập kỷ thiếu sáng tạo và suy giảm doanh số sách in. Các bộ
truyện siêu anh hùng, vốn trước đây chỉ phổ biến ở Hoa Kỳ, nay đang trên đường
trở thành một phần của văn hóa đại chúng như vị thế của manga. Có thể nói, văn
hóa truyện tranh nói chung đang ở thời kì phát triển mạnh, xét theo tiêu chí về
độ phổ biến và thái độ đón nhận của xã hội.

Hiện nay, hầu hết mọi người đọc truyện qua các trang web tổng hợp truyện tranh.
Những trang web này có hai ưu điểm chính:

\begin{itemize}
    \item
        Số lượng: Mỗi trang cung cấp ít nhất hàng nghìn đầu truyện.
    \item
        Tốc độ: Tốc độ ra truyện rất nhanh. Với các bộ truyện nổi tiếng, thường
        chỉ trong vòng một vài giờ sau khi ra mắt, chương mới đã xuất hiện.
\end{itemize}

Tuy vậy, nhược điểm chính của những trang web này là chất lượng ảnh của truyện.
Để giảm thời gian tải và tránh tốn băng thông, hình ảnh của truyện thường được
nén khá nhiều, gây vỡ hình, mờ nhòe. Một bộ phận người đọc, hoặc kĩ tính, hoặc
muốn sưu tầm truyện, thường chọn đọc những tệp truyện chất lượng cao, thường có
đuôi CBZ hoặc CBR. Bản chất tệp truyện này là các tệp nén zip bình thường, bên
trong có các tệp ảnh thông dụng như JPG, PNG. Tuy nhiên, do được tải hẳn về máy
rồi mới đọc, những tệp truyện này không bị giới hạn về băng thông hay thời gian,
do đó hình ảnh trong tệp có thể có chất lượng rất cao.

Trong khóa luận này, tôi viết một ứng dụng Android nhằm phục vụ số ít người dùng
có nhu cầu đọc truyện tranh chất lượng cao đã giới thiệu ở trên. Tên của ứng
dụng là \emph{yacv}, viết tắt của cụm từ tiếng Anh ``Yet Another Comic Viewer'',
tạm dịch là ``Lại một ứng dụng xem truyện tranh nữa''. Hai tính năng chính duy
nhất của ứng dụng là đọc và quản lí cơ bản (tìm kiếm, xóa) tệp truyện tranh có
sẵn trên điện thoại.

Cần chú ý rằng ứng dụng yacv chỉ bao gồm các tính năng liên quan đến đọc truyện
ngoại tuyến, đọc các tệp truyện có sẵn trên điện thoại người dùng. Ứng dụng
không phải là ứng dụng khách cho các trang đọc truyện hiện có, hay có máy chủ
tập trung riêng để cung cấp truyện.


%----------------------------------------------------------------------------------------
%	1.2: Ứng dụng tương tự
%----------------------------------------------------------------------------------------

\section{Ứng dụng tương tự}\label{sec:similar-apps}


% Links
\newcommand{\ComicScreen}{https://play.google.com/store/apps/details?id=com.viewer.comicscreen\&hl=en\&gl=US}
\newcommand{\AstonishComic}{https://play.google.com/store/apps/details?id=com.aerilys.acr.android\&hl=en\&gl=US}
\newcommand{\Tachiyomi}{https://github.com/tachiyomiorg/tachiyomi}
\newcommand{\TachiyomiIssue}{https://github.com/tachiyomiorg/tachiyomi/issues/1745}
\newcommand{\KuroReader}{https://play.google.com/store/apps/details?id=br.com.kurotoshiro.leitor_manga\&hl=en\&gl=US}


Hiện có nhiều ứng dụng đọc truyện tranh ngoại tuyến như yacv trên chợ ứng dụng
Google Play. Hai ứng dụng phổ biến nhất trong số này là
\href{\ComicScreen}{ComicScreen} và \href{\AstonishComic}{Astonishing Comic
Reader}. ComicScreen là ứng dụng có nhiều người dùng hơn. Các tính năng của
ComicScreen giống với các tính năng của yacv, tuy nhiên ComicScreen có thêm
nhiều chức năng phụ, đáng kể nhất là khả năng đọc từ mạng FTP/SMB và khả năng
sửa ảnh trong file. Astonishing Comic Reader cũng có chức năng tương tự yacv,
không hơn, tuy nhiên giao diện khá trau chuốt. Cả hai đều miễn phí và có quảng
cáo, được cập nhật có thể nói là thường xuyên.

Một ngoại lệ đáng kể ở đây là ứng dụng mã nguồn mở \href{\Tachiyomi}{Tachiyomi}.
Ứng dụng này có hệ thống phần mở rộng, cho phép đọc truyện ở các trang web
truyện tranh. Khi web truyện tranh thay đổi, hoặc hỗ trợ thêm trang mới, chỉ cần
tải về phần mở rộng tương ứng ở dạng ứng dụng APK. Tính năng này cùng mô hình mã
nguồn mở khiến Tachiyomi mạnh hơn, cập nhật nhanh hơn toàn bộ các ứng dụng đã có
và sẽ có. Tuy nhiên, Tachiyomi lại không thể được đưa lên Play Store, vì chính
tính năng phần mở rộng đã \href{\TachiyomiIssue}{vi phạm chính sách} của Play
Store \cite{GH_TACHI}.

Một điểm khác biệt quan trọng của yacv với các ứng dụng có sẵn là việc hỗ trợ
metadata của tệp truyện tranh, do các ứng dụng có sẵn trên Play Store đa số bỏ
qua thông tin này trong tệp truyện. Một trong số rất ít những ứng dụng hỗ trợ
tính năng này là \href{\KuroReader}{Kuro Reader}, tuy nhiên đây là một tính năng
trả phí.


%----------------------------------------------------------------------------------------
%	1.3: Kết quả đạt được
%----------------------------------------------------------------------------------------

\section{Kết quả đạt được}\label{sec:resulted-app}

Ứng dụng có các tính năng đủ dùng theo mục đích đã đề ra:

\begin{itemize}
    \item
        Đọc file truyện CBZ
    \item
        Tìm kiếm truyện theo metadata
\end{itemize}

Tính năng đọc tệp truyện CBR hiện mới chỉ được cài đặt một phần, do khó khăn
trong việc tích hợp thư viện đọc định dạng này.


%----------------------------------------------------------------------------------------
%	1.4: Cấu trúc khóa luận
%----------------------------------------------------------------------------------------

\section{Cấu trúc khóa luận}\label{sec:outline}

Các phần còn lại của khóa luận có cấu trúc như sau:

\begin{itemize}
    \item
        \fullref{chap:fundamental}: Giới thiệu sơ lược về ba nền tảng của ứng
        dụng, gồm hệ điều hành Android, ngôn ngữ lập trình Kotlin, và mẫu thiết
        kế MVVM; định dạng tệp nén ZIP cũng được giới thiệu vì liên quan trực
        tiếp đến ứng dụng.
    \item
        \fullref{chap:requirements}: Phân tích nhu cầu và ca sử dụng để có đặc
        tả yêu cầu.
    \item
        \fullref{chap:design}: Thiết kế ứng dụng, gồm thiết kế cơ sở dữ liệu,
        giao diện, logic nghiệp vụ.
    \item
        \fullref{chap:implementation}: Một số cài đặt và ca kiểm thử trong ứng
        dụng sẽ được nêu một cách có chọn lọc.
    \item
        \fullref{chap:conclusion}: Kết thúc khóa luận.
\end{itemize}

Sau phần Tài liệu Tham khảo, khóa luận có hai phụ lục về metadata của truyện
tranh.

\end{document}
