\providecommand{\main}{../..}

\documentclass[../../thesis]{subfiles}


\begin{document}

\chapter{Phân tích yêu cầu}\label{chap:requirements}

Chương này tìm hiểu đối tượng người dùng nhắm tới để tìm ra \emph{Nhu cầu} của
họ. Nhu cầu này sau đó được phân tích thành các \emph{Yêu cầu chức năng} và
\emph{phi chức năng}. Cuối cùng, các chức năng được mổ xẻ, mô tả kĩ để được
\emph{Đặc tả ca sử dụng}, là bộ khung cho quá trình phát triển ứng dụng.


%----------------------------------------------------------------------------------------
%	3.1: Mô tả chung
%----------------------------------------------------------------------------------------

\section{Mô tả chung}\label{sec:overview}

\subsection{Người dùng}\label{sec:users}

Ứng dụng yacv tập trung vào một số ít người dùng, là một trong hai nhóm sau:

\begin{itemize}
    \item
        Người dùng sưu tầm truyện
    \item
        Người dùng có yêu cầu đọc truyện với chất lượng hình ảnh cao
\end{itemize}

Cả hai nhóm có điểm chung là kĩ tính, yêu cầu cao về trải nghiệm đọc truyện, cụ
thể là về \emph{chất lượng hình ảnh}. Cũng do kĩ tính, nên cả hai nhóm không cần
nhiều chức năng, tuy nhiên từng chức năng cần hoàn thiện. Nhóm người dùng sưu
tầm truyện còn có yêu cầu \emph{xem thông tin (metadata)} của tệp truyện.

Tóm lại, ta thu được hai \emph{Nhu cầu} của nhóm người dùng hướng đến:

\begin{itemize}
    \item Đọc truyện trong tệp truyện
    \item Xem metadata
\end{itemize}

\subsection{Mục đích}\label{sec:objectives}

Trước khi đi vào chi tiết yêu cầu ở mục tiếp theo, tôi muốn làm rõ mục đích của
sản phẩm đã nhắc ở \autoref{sec:background}.

\begin{itemize}
    \item
        Ứng dụng yacv chỉ bao gồm các tính năng liên quan đến đọc \textbf{truyện
        tranh} và là ứng dụng \textbf{ngoại tuyến} (tức đọc các tệp truyện có
        sẵn trên điện thoại).
    \item
        Ứng dụng \emph{không phải} là ứng dụng khách cho các trang đọc truyện
        hiện có, hay có máy chủ tập trung riêng để cung cấp truyện.
    \item
        Ứng dụng \emph{không có} khả năng đọc truyện đuôi PDF, cùng với các định
        dạng truyện thiên về chữ khác như TXT, EPUB.
\end{itemize}

Các giới hạn này nhằm tránh cho phần mềm quá phức tạp với tôi, đồng thời phù hợp
(không thừa thiếu chức năng) so với nhu cầu của nhóm người dùng mục tiêu đã nêu
ở \autoref{sec:users}.


%----------------------------------------------------------------------------------------
%	3.2: Yêu cầu đặt ra
%----------------------------------------------------------------------------------------

\section{Yêu cầu đặt ra}\label{sec:requirements}

\subsection{Yêu cầu chức năng}\label{sec:functional-requirements}

Ứng dụng có các chức năng chính sau:

\begin{itemize}
    \item
        Quét các tệp truyện trên thiết bị
    \item
        Hiển thị danh sách truyện
    \item
        Đọc truyện
    \item
        Xem metadata truyện
    \item
        Tìm kiếm truyện
    \item
        Xóa truyện
\end{itemize}

\subsection{Yêu cầu phi chức năng}\label{sec:non-functional-requirements}

Ứng dụng cần đạt một số tiêu chí sau:

\begin{itemize}
    \item
        Phản hồi nhanh: Các thao tác cần có thời gian phản hồi nhanh. Phản hồi
        nhanh không nhất thiết là thời gian thực thi ngắn, mà là luôn có các
        thông báo tiến độ cho người dùng.

        \begin{itemize}
            \item
                Luôn hiện thông báo chờ khi làm việc gì đó lâu
            \item
                Nếu có nhiều kết quả tìm kiếm, hiển thị từ từ, đưa kết quả đã
                biết lên trước
        \end{itemize}
    \item
        Tốc độ xem truyện chấp nhận được: Đây là một phần của phản hồi nhanh,
        nhưng được tách riêng vì độ quan trọng của nó. Tốc độ xem truyện gồm hai
        tiêu chí:

        \begin{itemize}
            \item
                Tốc độ mở truyện, tức tốc độ xem trang đầu (có thể so với first
                contentful paint trong lập trình web)
            \item
                Tốc độ cuộn trang tới-lui
        \end{itemize}
    \item
        Chiếm ít bộ nhớ: Bộ nhớ chiếm dụng của ứng dụng gồm hai phần: bộ nhớ RAM
        và bộ nhớ tạm, cả hai cần sử dụng ít dung lượng nhất có thể. Đây là một
        yêu cầu đáng cân nhắc, lí do vì kích cỡ từng tệp truyện thường rất lớn
        (từ vài chục đến hơn một trăm megabyte), tuy nhiên cần chú ý cân bằng
        yêu cầu này với yêu cầu về tốc độ (đánh đổi không gian-thời gian).
    \item
        Giao diện đơn giản, trực quan: Người dùng hướng đến có thể xếp vào nhóm
        người dùng ``say mê'' (enthusiast), do đó giao diện chỉ cần đơn giản rõ
        ràng, không màu mè, tập trung vào tính năng.
\end{itemize}


%----------------------------------------------------------------------------------------
%	3.2: Phân tích yêu cầu
%----------------------------------------------------------------------------------------

\section{Phân tích yêu cầu}\label{sec:analysis}

Mỗi yêu cầu đã xác định trong \autoref{sec:functional-requirements} được coi là
một ca sử dụng, được trình bày trong các tiểu mục dưới đây.

\emph{Người dùng duy nhất} trong các ca sử dụng là \emph{người đọc}, do đó hai
cụm từ này sẽ được dùng hoán đổi cho nhau. Do ứng dụng hoàn toàn ngoại tuyến,
người đọc cũng không có tương tác với nhau.

Do ứng dụng đơn giản, các ca sử dụng tách biệt, nên mỗi ca sử dụng gắn với một
\emph{màn hình}. Có tổng cộng năm màn hình sẽ được mô tả, gồm:

\begin{itemize}
    \item
        Màn hình Thư viện
    \item
        Màn hình Thư mục
    \item
        Màn hình Đọc truyện
    \item
        Màn hình Metadata
    \item
        Màn hình Tìm kiếm
\end{itemize}

\subsection{Quét các tệp truyện trên thiết bị}\label{sec:scan}

\begin{itemize}
    \item
        \textbf{Mô tả}:

        Người đọc \emph{chọn} một thư mục trong điện thoại làm thư mục gốc. Ứng
        dụng sẽ \emph{quét} thư mục này và tìm các tệp truyện, rồi hiển thị
        những thư mục chứa tệp truyện cho người đọc chọn.
    \item
        \textbf{Luồng chính}:

        \begin{enumerate}
            \item
                Người đọc bật ứng dụng (tức ở Màn hình Thư viện)
            \item
                Người đọc ấn vào nút thay đổi thư mục gốc.
            \item
                Trình chọn thư mục của Android hiện ra, cho phép người đọc chọn
                thư mục làm thư mục gốc.
            \item
                Màn hình Thư viện trở lại, quét và hiển thị các thư mục chứa
                truyện trong thư mục gốc.
        \end{enumerate}
    \item
        \textbf{Luồng thay thế}:

        Nếu người đọc đã chọn một thư mục gốc, ca sử dụng này \emph{thay thế}
        thư mục gốc đã chọn bằng thư mục vừa chọn.

        Nếu người đọc không chọn thư mục nào, quay lại Màn hình Thư viện và
        không thay đổi gì.
    \item
        \textbf{Luồng ngoại lệ}:

        Nếu có lỗi trong quá trình chọn thư mục, cần gợi ý người đọc chọn lại.
        Lỗi gồm:

        \begin{itemize}
            \item
                Thiếu quyền đọc
            \item
                Không tìm được thư mục gốc
            \item
                Thư mục gốc không có truyện
        \end{itemize}

        Nếu có lỗi trong quá trình quét cần phải giảm thiểu và giấu khỏi người
        đọc.
    \item
        \textbf{Điều kiện}:

        \begin{itemize}
            \item
                Tiền điều kiện: Ứng dụng ở Màn hình Thư viện
            \item
                Hậu điều kiện: Ứng dụng ở Màn hình Thư viện

                \begin{itemize}
                    \item
                        Hiển thị thư mục truyện quét được
                    \item
                        Hiển thị lỗi nếu có (ba loại lỗi ở trên), và gợi ý xử lí
                \end{itemize}
          \end{itemize}
    \item
        \textbf{Yêu cầu phi chức năng}:

        Nếu đang quét, Màn hình Thư viện cần hiển thị danh sách thư mục theo
        tiến độ, ứng dụng quét đến đâu hiển thị đến đấy.

        Mỗi thư mục cần hiển thị:

        \begin{itemize}
            \item
                Tên thư mục
            \item
                Ảnh đại diện cho thư mục: bìa một truyện bất kì tìm được trong
                thư mục
        \end{itemize}
\end{itemize}

Đây là ca sử dụng đầu tiên khi người đọc chạy ứng dụng lần đầu. Các tệp truyện
sẽ được quét từ thư mục gốc, rồi được gom lại theo thư mục như mô tả ở
\hyperref[sec:show-library]{ca sử dụng kế tiếp}.

Màn hình đầu tiên khi người đọc bật lên gọi là \emph{Màn hình Thư viện} (Library
screen). Các thư mục chứa truyện, hoặc thông báo lỗi liên quan đến bản thân quá
trình chọn truyện (đã miêu tả trong bước 6 ở trên) sẽ được hiển thị ở màn hình
này.

Khi quét, ứng dụng phải đọc luôn cả metadata của tệp truyện nếu có. Các trường
trong metadata được giải thích chi tiết trong \autoref{app:metadata}. Hiện nay,
yacv chấp nhận định dạng metadata ComicInfo, là một tệp XML trong tệp truyện.
\autoref{app:comic-info-xsd} trình bày lược đồ XSD của định dạng metadata này.

Một số metadata có thể được trích xuất ngay từ tên tệp truyện. Do không có quy
chuẩn trong việc đặt tên tệp, cách trích xuất này không ổn định, tuy nhiên cũng
không phải ý tưởng tồi. Nếu có thể, dựa vào \autoref{app:metadata} để thử trích
xuất từ tên tệp truyện.

Tới đây người đọc có thể thực hiện các ca sử dụng khác, trong đó quan trọng nhất
là duyệt theo thư mục rồi xem truyện.

\subsection{Hiển thị danh sách truyện}\label{sec:show-library}

\begin{itemize}
    \item
        \textbf{Mô tả}:

        Người đọc duyệt truyện theo thư mục, rồi chọn truyện và xem.
    \item
        \textbf{Luồng chính}:

        \begin{enumerate}
            \item
                Người đọc bật ứng dụng (tức ở Màn hình Thư viện), có danh sách
                truyện (tức đã chọn thư mục gốc và quét được ít nhất một thư mục
                chứa truyện).
            \item
                Người đọc chọn một thư mục.
            \item
                Ứng dụng chuyển sang Màn hình Thư mục, hiển thị \emph{danh sách
                truyện} trong thư mục đó cho người đọc xem và chọn.
        \end{enumerate}
    \item
        \textbf{Điều kiện}:

        \begin{itemize}
            \item
                Tiền điều kiện:

                \begin{itemize}
                    \item
                        Ứng dụng ở Màn hình Thư viện
                    \item
                        Đã chọn thư mục gốc và đã quét được ít nhất một thư mục
                        chứa truyện
                \end{itemize}
            \item
                Hậu điều kiện: Ứng dụng ở Màn hình Thư mục.
        \end{itemize}
    \item
        \textbf{Yêu cầu phi chức năng}:

        Nếu đang quét, màn hình Thư mục cần hiển thị danh sách truyện theo tiến
        độ, ứng dụng quét đến đâu hiển thị đến đấy.

        Mỗi truyện cần hiển thị:

        \begin{itemize}
            \item
                Tên truyện
            \item
                Bìa truyện
            \item
                Tiến độ đọc
            \item
                Đánh giá yêu thích
        \end{itemize}
  \end{itemize}

Đây là một trong hai ca sử dụng chính của ứng dụng, bên cạnh (và là tiền điều
kiện cho) \hyperref[sec:read-comic]{ca sử dụng đọc truyện} sẽ được miêu tả tiếp
theo.

Màn hình khi người đọc chọn một thư mục gọi là \emph{Màn hình Thư mục}
(Directory screen). Cũng giống như Màn hình Thư viện, ảnh bìa và tên của truyện
được hiển thị để người đọc chọn.

Trong yacv, truyện được quản lí và duyệt theo thư mục. Có hai lí do cho lựa chọn
thiết kế này:

\begin{itemize}
    \item
        Giảm độ phức tạp khi lập trình
    \item
        Các phương pháp duyệt khác không trực quan

        \begin{itemize}
            \item
                Các phương pháp duyệt khác chỉ bao gồm duyệt theo metadata, tức
                duyệt theo các thông tin đi kèm như Tác giả, Nhân vật, Bộ
                truyện,\ldots{} thì yêu cầu truyện phải có đủ metadata. Trên
                thực tế, không phải tệp truyện nào cũng có đủ thông tin này, do
                vậy sẽ có trường hợp rất nhiều truyện bị gom vào mục ``Không đủ
                thông tin''. Hơn nữa, giả sử truyện có đi kèm metadata, ta xem
                xét tiếp trường hợp dưới.
            \item
                Giả sử ta quản lí theo Nhân vật: Vậy để trực quan, yacv phải
                hiển thị ảnh nhân vật. Hiện nay, việc nhận diện và cắt đúng ảnh
                phần mặt nhân vật ra để tạo ảnh đại diện có thể nói là bất khả
                thi. Do vậy, khi duyệt theo Nhân vật, người đọc chỉ có thể thấy
                tên, không thấy một hình ảnh gợi ý nào khác, dẫn đến khó khăn
                khi sử dụng. Lập luận tương tự có thể dùng với các cách xếp
                khác.
            \item
                Một cách xếp có thể nói là tốt là xếp theo Bộ truyện, tuy nhiên
                ta lại quay về vấn đề thiếu metada.
        \end{itemize}
\end{itemize}

Hơn nữa, các thư mục cần được ``làm phẳng'', tức là hiển thị thư mục con
(cháu,\ldots) ngang hàng với thư mục gốc. Bảng sau cho thấy cách yacv làm phẳng
cây thư mục:

% 1st problem: verbatim inside table
% 2nd problem: font for those lines
% | Cây thư mục gốc                   | yacv đã làm phẳng         |
% |-----------------------------------|---------------------------|
% | thư mục gốc                       | thư mục gốc               |
% | ├── Original Sin #1.cbz           | └── Original Sin #1.cbz   |
% | └── House of M                    | House of M                |
% |     ├── House of M #1.cbz         | ├── House of M #1.cbz     |
% |     ├── House of M #3.cbz         | └── House of M #3.cbz     |
% |     └── Tie-ins                   | Tie-ins                   |
% |         └── Black Panther #7.cbz  | └── Black Panther #7.cbz  |
% Bảng 2: Cách yacv làm phẳng thư mục

\begin{table}[H]
    \centering
    \caption{Cách yacv làm phẳng thư mục}
    \label{tab:yacv-flattening}
    \begin{tabular}{ll}
        \toprule
        Cây thư mục gốc & yacv đã làm phẳng \\
        \midrule
        % See preamble of listings for the %* *)
        \begin{lstlisting}
%*thư mục gốc*)
    Original Sin #1.cbz
    House of M
        House of M #1.cbz
        House of M #3.cbz
        Tie-ins
            Black Panther #7.cbz
        \end{lstlisting}
        &
        \begin{lstlisting}
%*thư mục gốc*)
    Original Sin #1.cbz
House of M
    House of M #1.cbz
    House of M #3.cbz
Tie-ins
    Black Panther #7.cbz
        \end{lstlisting}
        \\
        \bottomrule
    \end{tabular}
\end{table}

Theo như bảng trên, các màn hình trong yacv được tổ chức như sau:

\begin{itemize}
    \item
        Màn hình Thư viện: có 3 thư mục:

        \begin{itemize}
            \item
                thư mục gốc
            \item
                House of M
            \item
                Tie-ins
        \end{itemize}
    \item
        Khi chọn ``House of M'': chuyển sang Màn hình Thư mục tương ứng, không
        có thư mục con, và có 2 tệp truyện:

        \begin{itemize}
            \item
                House of M \#1.cbz
            \item
                House of M \#3.cbz
        \end{itemize}
    \item
        Tương tự với các thư mục khác.
\end{itemize}

Có ba lí do cho lựa chọn thiết kế này:

\begin{itemize}
    \item
        Giảm độ phức tạp khi lập trình.
    \item
        Người đọc không phải đi qua nhiều tầng thư mục để đến được tệp truyện
        cần đọc.
    \item
        Không có ca sử dụng có ý nghĩa cho thư mục lồng nhau:

        Trường hợp hợp lí nhất cho việc có thư mục lồng nhau là khi lưu các tệp
        truyện liên quan đến một bộ truyện (tie-ins), như cột trái Bảng 3:

        \begin{itemize}
            \item
                Thư mục cha (House of M) chứa tệp truyện trong bộ truyện cùng
                tên và thư mục tie-ins.
            \item
                Thư mục Tie-ins chứa các tệp truyện tie-in.
        \end{itemize}

        Tuy nhiên, bản thân các tệp tie-in lại là tệp truyện thông thường trong
        một bộ truyện khác, do đó nếu tổ chức thư mục như thế này sẽ dẫn đến
        tình trạng lặp tệp truyện, là điều không mong muốn ngay cả với máy tính.
\end{itemize}

\subsection{Đọc truyện}\label{sec:read-comic}

\begin{itemize}
    \item
        \textbf{Mô tả}:

        Người đọc chọn một truyện để xem.
    \item
        \textbf{Luồng chính}:

        \begin{enumerate}
            \item
                Người đọc bật ứng dụng, đã chọn thư mục gốc, đã quét được ít
                nhất một thư mục chứa truyện, đã chọn một thư mục (tức ở Màn
                hình Thư mục).
            \item
                Ứng dụng hiển thị danh sách truyện trong thư mục đó cho người
                đọc xem và chọn.
            \item
                Người đọc chọn một truyện và đọc.
            \item
                Màn hình Đọc truyện hiển thị trang truyện cho người đọc.
            \item
                Người đọc vuốt qua lại theo phương ngang để chuyển trang.
        \end{enumerate}
    \item
        \textbf{Luồng thay thế}:

        Xem phần Màn hình Tìm kiếm. Màn hình Đọc truyện có thể được kích hoạt
        bằng cách ấn vào truyện hiển thị trong màn hình này.
    \item
        \textbf{Luồng ngoại lệ}:

        Nếu tệp truyện không tìm thấy được, báo cho người đọc và giữ nguyên ở
        Màn hình Thư mục.
    \item
        \textbf{Điều kiện}:

        \begin{itemize}
            \item
                Tiền điều kiện: Ứng dụng ở Màn hình Thư mục.
            \item
                Hậu điều kiện: Ứng dụng ở Màn hình Đọc truyện.
        \end{itemize}
    \item
        \textbf{Yêu cầu phi chức năng}:

        \begin{itemize}
            \item
                Nếu người đọc đã đọc truyện, ứng dụng cần đưa về chính trang
                truyện đang đọc dở. Nếu đã đọc đến trang cuối, tức đã đọc xong,
                ứng dụng cần đưa về trang đầu tiên.
            \item
                Trải nghiệm cuộn trang mượt mà nhất có thể.
        \end{itemize}
  \end{itemize}

Đây là một trong hai ca sử dụng chính của ứng dụng, bên cạnh (và là mục đích
của) \hyperref[sec:show-library]{ca sử dụng hiển thị danh sách truyện} đã được
miêu tả ở trên.

Màn hình khi người đọc đọc một truyện gọi là \emph{Màn hình Đọc truyện}. Màn
hình này cho phép người đọc duyệt các trang truyện theo phương ngang. Mục tiêu
là thiết kế màn hình này sao cho có trải nghiệm gần giống nhất với ứng dụng Thư
viện ảnh (Gallery) tích hợp trong mọi điện thoại Android.

\subsection{Xem metadata truyện}\label{sec:view-metadata}

\begin{itemize}
    \item
        \textbf{Mô tả}:

        Trong Màn hình Đọc truyện, người đọc ấn nút để xem metadata.
    \item
        \textbf{Luồng chính}:

        \begin{enumerate}
            \item
                Người đọc bật ứng dụng, chọn một truyện để vào đến Màn hình Đọc
                truyện.
            \item
                Người đọc ấn nút Xem metadata.
            \item
                Ứng dụng hiển thị mọi metadata, bao gồm cả những trường bị
                thiếu. Ảnh bìa của truyện cũng được hiển thị kèm.
            \item
                Người dùng có thể đánh giá truyện bằng nút Yêu thích trong màn
                hình này, hoặc ngược lại (bỏ đánh giá Yêu thích).
        \end{enumerate}
    \item
        \textbf{Điều kiện}:

        \begin{itemize}
            \item
                Tiền điều kiện: Ứng dụng ở Màn hình Đọc truyện.
            \item
                Hậu điều kiện: Ứng dụng ở Màn hình Metadata.
        \end{itemize}
    \item
        \textbf{Yêu cầu phi chức năng}:

        \begin{itemize}
            \item
                Màn hình Metadata phải hiển thị ảnh bìa, cùng các metadata của
                truyện
            \item
                Những trường metadata trống phải ghi rõ ``Trống'',
                ``Unknown'',\ldots{}
        \end{itemize}
  \end{itemize}

Đây là một ca sử dụng phụ, có thể được kích hoạt khi người dùng đang ở Màn hình
Đọc truyện.

Màn hình khi người đọc xem metadata gọi là \emph{Màn hình Metadata}. Màn hình
này có thể có chức năng sửa metadata, tùy theo tiến độ khóa luận để xem xét có
cài đặt không.

Hệ thống đánh giá của ứng dụng chỉ ở mức cơ bản, gồm duy nhất tính năng Yêu
thích. Tính năng này cũng chỉ phục vụ hai mục đích là thể hiện sự đánh giá của
người dùng và lọc nhanh truyện về mặt thị giác (đã nhắc đến trong phần Mô tả
từng bước của \hyperref[sec:show-library]{ca sử dụng hiển thị danh sách
truyện}).

Các tính năng nâng cao hơn như gợi ý không xuất hiện, do một số lí do sau:

\begin{itemize}
    \item
        Giảm độ phức tạp khi lập trình.
    \item
        Người dùng không có nhu cầu: nhóm người dùng hướng đến có đặc điểm hiểu
        biết về truyện tranh, do đó việc gợi ý có thể coi là thừa thãi.
    \item
        Thiếu thông tin gợi ý: việc gợi ý chỉ có hiệu quả khi có một cơ sở dữ
        liệu về các bộ truyện liên quan, hoặc lựa chọn các truyện liên quan của
        cộng đồng người đọc, trong khi yacv là một ứng dụng hoàn toàn ngoại
        tuyến.
\end{itemize}

\subsection{Tìm kiếm truyện}\label{sec:search-comic}

\begin{itemize}
    \item
        \textbf{Mô tả}:

        Trong Màn hình Thư viện, người đọc ấn nút Tìm kiếm để tìm truyện.
    \item
        \textbf{Luồng chính}:

        \begin{enumerate}
            \item
                Người đọc bật ứng dụng.
            \item
                Người đọc ấn nút Tìm kiếm, gõ từ khóa cần tìm, và ấn nút Enter.
            \item
                Ứng dụng hiển thị kết quả tìm kiếm theo metadata và tên tệp
                truyện.
        \end{enumerate}
    \item
        \textbf{Luồng ngoại lệ}:

        Nếu không tìm thấy truyện, ứng dụng cần thông báo ở Màn hình Tìm kiếm.
    \item
        \textbf{Điều kiện}:

        \begin{itemize}
            \item
                Tiền điều kiện: Ứng dụng ở Màn hình Thư viện.
            \item
                Hậu điều kiện: Ứng dụng ở Màn hình Tìm kiếm.
        \end{itemize}
    \item
        \textbf{Yêu cầu phi chức năng}:

        \begin{itemize}
            \item
                Kết quả tìm kiểm cần được gom theo nhóm dựa vào trường metadata
                tìm thấy được. Nếu không có kết quả, phải báo cho người dùng.
            \item
                Nếu có thể, hiển thị ảnh bìa của truyện.
        \end{itemize}
  \end{itemize}

Đây là ca sử dụng phụ, có thể được kích hoạt khi người dùng ở Màn hình Thư viện.

Màn hình khi người đọc \emph{xem kết quả tìm kiếm} gọi là \emph{Màn hình Tìm
kiếm}. Màn hình này chỉ hiện ra khi người dùng ấn nút Enter để chính thức tìm
kiếm; cho đến trước lúc đó, ứng dụng vẫn ở Màn hình Thư viện.

\begin{wrapfigure}[8]{r}{4.5cm}
\vspace*{-10mm}
\begin{verbatim}
Truyện
- Watchmen #1.cbz
- Watchmen #2.cbz
Bộ truyện
- Watchmen
\end{verbatim}
\caption{Mô tả Màn hình Tìm kiếm}
\label{fig:search-screen-mockup}
\end{wrapfigure}

Màn hình Tìm kiếm nhóm kết quả theo trường metadata mà kết quả được tìm thấy. Ví
dụ, người dùng tìm ``Watchmen'' sẽ thấy Màn hình Tìm kiếm gần như
\autoref{fig:search-screen-mockup}.

Tương tác của người đọc với màn hình trên là như sau:

\begin{itemize}
    \item
        Khi ấn vào một mục trong danh sách ``Truyện'', người đọc được đưa đến
        thẳng Màn hình Đọc truyện của truyện đó (và hiển thị ở trang đọc dở như
        đã mô tả trong \hyperref[sec:read-comic]{ca sử dụng đọc truyện}).
\end{itemize}

% Split itemize to avoid confusing wrapfigure
% https://tex.stackexchange.com/a/285760/206774
\begin{itemize}[resume, before = \vspace*{-\dimexpr\topsep+\partopsep\relax}]
    \item
        Khi ấn vào một mục trong danh sách ``Bộ truyện'', người đọc được đưa đến
        màn hình chứa danh sách truyện trong bộ truyện đã chọn. \emph{Màn hình
        này cần giống với Màn hình Thư mục}. Sau đó, người dùng chọn một truyện
        để đọc như bình thường.
\end{itemize}

Đây chỉ là ví dụ về một từ khóa có kết quả khi tìm theo tên tệp truyện và bộ
truyện. Các trường metadata khác nếu có kết quả phù hợp cũng sẽ thể hiện theo
hình thức trên.

Chú ý rằng bìa truyện luôn được thể hiện khi có thể. Trong ví dụ trên, chắc chắn
phải có bìa truyện cho mọi mục con trong danh sách ``Truyện''. Còn mục con của
``Bộ truyện'' thì không, lí do là không có đủ dữ liệu (không có dữ liệu cho
logo, banner,.. của bộ truyện). Tương ứng, không có dữ liệu hiển thị cho danh
sách ``Nhân vật'', ``Tác giả'',\ldots{} Đây là một hạn chế quan trọng về giao
diện mà hiện chưa có cách thiết kế hợp lý.

Nếu không có kết quả, cần thể hiện rõ cho người dùng biết.

Khi người dùng gõ từ khóa để tìm kiếm, một số thông tin gợi ý tìm kiếm (từ khóa
cũ, từ khóa liên quan) có thể hiện ra; tính năng này tùy theo tiến độ khóa luận
để xem xét có cài đặt không. Chỉ khi người dùng ấn Enter, quá trình tìm kiếm mới
bắt đầu, và hiển thị kết quả, chứ không hiển thị kết quả theo quá trình người
dùng gõ phím. Lí do cho lựa chọn này như sau:

\begin{itemize}
    \item
        Giảm độ phức tạp khi lập trình.
    \item
        Không quá cần thiết: ví dụ, bộ máy tìm kiếm của Google cũng chỉ hiện gợi
        ý khi người dùng nhập từ khóa tìm kiếm, phải đến khi ấn Enter thì quá
        trình tìm kiếm mới diễn ra và kết quả chi tiết được hiển thị.
\end{itemize}

Với độ phức tạp dự kiến của việc hiển thị ảnh bìa truyện, đây có thể được xem là
đánh đổi hợp lí để tăng hiệu năng ứng dụng.

\subsection{Xóa truyện}\label{sec:delete-comic}

\begin{itemize}
    \item
        \textbf{Mô tả}:

        Người dùng chọn một số truyện trong một màn hình chứa danh sách truyện
        để xóa.
    \item
        \textbf{Luồng chính}:

        \begin{enumerate}
            \item
                Người dùng truy cập vào một màn hình chứa danh sách truyện (là
                Màn hình Thư mục hoặc Màn hình Tìm kiếm).
            \item
                Người dùng ấn và giữ vào một truyện.
            \item
                Màn hình đó sẽ chuyển sang chế độ xóa, báo hiệu bằng biểu tượng
                Thùng rác trên màn hình, và ô đánh dấu để xóa ở cạnh mỗi truyện.
                Truyện mà người dùng ấn giữ phải được đánh dấu xóa ngay.
            \item
                Người dùng có thể chọn thêm truyện để xóa nếu muốn.
            \item
                Người dùng ấn nút xóa để xóa truyện.
            \item
                Ứng dụng hiện ra hộp thoại xóa, ghi rõ rằng truyện sẽ được xóa
                khỏi bộ nhớ điện thoại, số truyện sẽ xóa, và hỏi người dùng có
                thực sự muốn xóa không.
            \item
                Nếu người dùng ấn vào nút Đồng ý xóa, truyện sẽ được xóa khỏi bộ
                nhớ điện thoại và tắt hộp thoại, nếu không thì tắt hộp thoại.
            \item
                Sau khi tắt hộp thoại, màn hình trở về chế độ ban đầu, biểu
                tượng thùng rác cũng biến mất, truyện được xóa cũng biến mất.
        \end{enumerate}
    \item
        \textbf{Điều kiện}

        Tiền và hậu điều kiện đều là màn hình chứa danh sách truyện, gồm:

        \begin{itemize}
            \item
                Màn hình Thư mục
            \item
                Màn hình Tìm kiếm
        \end{itemize}
\end{itemize}

Ca sử dụng này không có màn hình riêng biệt, mà sử dụng một chế độ của các màn
hình hiển thị danh sách truyện.

\end{document}
