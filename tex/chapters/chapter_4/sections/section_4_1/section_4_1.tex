\documentclass[../../../../thesis]{subfiles}

\IfEq{\jobname}{\detokenize{thesis}}{}{%
    \setcounterref{chapter}{chap:design}
}

\IfEq{\jobname}{\detokenize{thesis}}{}{%
    \StrBehind{\getrefnumber{sec:module-database}}{.}[\tempsectioncounter]
    \setcounter{section}{\tempsectioncounter}
    \addtocounter{section}{-1}
}

\IfEq{\jobname}{\detokenize{thesis}}{}{%
    \StrBehind{\getrefnumber{fig:db-schema}}{.}[\tempfigurecounter]
    \setcounter{figure}{\tempfigurecounter}
    \addtocounter{figure}{-1}
}


\begin{document}

\section{Module Database}\label{sec:module-database}

Thông thường mục này được tách riêng ra, xếp vào mục \emph{Thiết kế cơ sở dữ
liệu}, ngang hàng với mục Thiết kế hướng đối tượng. Tuy nhiên, yacv còn cần xử
lí dữ liệu khác quan trọng không kém là dữ liệu ảnh. Do không còn có vai trò
trung tâm, duy nhất, phần cơ sở dữ liệu chỉ được coi là một module trong thiết
kế hướng đối tượng của ứng dụng.

yacv chọn SQLite vì đây là một cơ sở dữ liệu gọn nhẹ nhúng sẵn trong Android.
SQLite sử dụng mô hình quan hệ, do đó thiết kế bảng cần đảm bảo được chuẩn hóa
(normalization).

Do không cần quản lí người dùng, cơ sở dữ liệu của yacv chỉ dùng để \emph{lưu
thông tin metadata}, cho phép ứng dụng quét dữ liệu ít lần hơn và tìm kiếm
truyện. Theo như yêu cầu về metadata ở hai Phụ lục, và sau khi chuẩn hóa, ta có
lược đồ cơ sở dữ liệu như sau:

Các bảng thực thể gồm:

\begin{itemize}
    \item
        \texttt{Comic}: lưu thông tin \emph{tập truyện lẻ}, là bảng trung tâm
    \item
        \texttt{Series}: lưu thông tin \emph{bộ truyện}
    \item
        \texttt{Author}: lưu tên tác giả
    \item
        \texttt{Role}: lưu vai trò của tác giả trong một tập truyện
    \item
        \texttt{Character}: lưu tên nhân vật
    \item
        \texttt{Genre}: lưu tên thể loại truyện
\end{itemize}

\begin{wrapfigure}[26]{r}{10cm}
    \vspace*{-10mm}
    \includesvg[width=\linewidth,inkscapelatex=false]
        {../images/relationships.real.large.svg}
    \vspace*{-10mm}
    \caption{Lược đồ cơ sở dữ liệu của yacv}
    \label{fig:db-schema}
\end{wrapfigure}

Một hạn chế quan trọng của các bảng \texttt{Character} và \texttt{Author} là
chúng chỉ lưu thông tin tên, và chỉ phân biệt với nhau bằng tên. Nếu có hai tác
giả/nhân vật trùng tên, yacv không thể phát hiện và hiển thị riêng.

Xét bảng trung tâm \texttt{Comic}. Bảng này có một số trường không phải metadata
mà dùng để lưu thông tin của riêng ứng dụng, gồm:

\begin{itemize}
    \item
        \texttt{CurrentPage}: lưu trang đang đọc
    \item
        \texttt{Love}: lưu trạng thái Yêu thích
    \item
        \texttt{ReadCount}: lưu số lần đọc
\end{itemize}

Trong lược đồ, có nhiều trường nhìn qua không cần thiết nhưng thực tế có ích, do
thư viện SAF đã mô tả ở \autoref{chap:fundamental}:

\begin{itemize}
    \item
        Trường \texttt{FileUri} trong \texttt{Comic}: Lưu đường dẫn của tệp
        truyện ở dạng URI.
    \item
        Trường \texttt{FolderUri} trong \texttt{Folder}: Lưu đường dẫn của thư
        mục ở dạng URI.
\end{itemize}

% Split itemize to avoid confusing wrapfigure
% https://tex.stackexchange.com/a/285760/206774
\begin{itemize}[resume, before = \vspace*{-\dimexpr\topsep+\partopsep\relax}]
    \item
        Trường \texttt{Name} trong \texttt{Folder}: Tên thư mục. Thông thường
        nếu có đường dẫn, có thể tìm ra tên thư mục rất nhanh, tuy nhiên cũng do
        SAF mà việc này trở nên khó khăn, nên cần lưu riêng trường này.
\end{itemize}

Các trường URI đều cần có ràng buộc \texttt{UNIQUE}, do mỗi URI trỏ đích danh
đến một đối tượng.

Ta xem xét đến các bảng nối:

\begin{itemize}
    \item
        \texttt{ComicCharacterJoin}:

        \begin{itemize}
            \item
                Mỗi tập truyện có thể có nhiều nhân vật và ngược lại, do đó
                \texttt{Comic} và \texttt{Character} có quan hệ Nhiều - Nhiều.
            \item
                Chú ý rằng các nhân vật có quan hệ với tập truyện chứ không phải
                bộ truyện, vì có nhân vật phụ (không xuất hiện trong mọi tập
                truyện).
        \end{itemize}
    \item
        \texttt{ComicAuthorJoin}:

        \begin{itemize}
            \item
                Mỗi tập truyện có thể có nhiều tác giả và ngược lại, do đó
                \texttt{Comic} và \texttt{Author} có quan hệ Nhiều - Nhiều.
            \item
                Chú ý rằng các tác giả có quan hệ với tập truyện chứ không phải
                bộ truyện, vì mô hình xuất bản nhiều truyện tranh là nhà xuất
                bản sở hữu nhân vật và thuê người viết.
            \item
                Đồng thời, một tác giả có thể giữ vai trò khác nhau trong các bộ
                truyện khác nhau, do đó bảng này còn nối với bảng \texttt{Role}.
        \end{itemize}
\end{itemize}

% Split itemize to avoid confusing wrapfigure
% https://tex.stackexchange.com/a/285760/206774
\begin{itemize}[resume, before = \vspace*{-\dimexpr\topsep+\partopsep\relax}]
    \item
        \texttt{ComicGenreJoin}: Mỗi tập truyện có thể có nhiều thể loại khác
        nhau và ngược lại, do đó \texttt{Comic} và \texttt{Genre} có quan hệ
        Nhiều - Nhiều.
\end{itemize}

Do dùng Room, mỗi bảng ứng với một lớp. Các truy vấn với bảng cần đóng gói dữ
liệu vào các lớp này, trước khi gửi đến hoặc nhận về từ \emph{DAO} (Data Access
Object). Mỗi câu lệnh lại được chuyển thành một hàm trong DAO.

\end{document}
